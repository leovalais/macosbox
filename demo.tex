\documentclass[a4paper]{article}
\usepackage[utf8]{inputenc}
\usepackage[scaled]{helvet}
\renewcommand\familydefault{\sfdefault}
\usepackage[T1]{fontenc}
\usepackage{minted}
\usepackage{macosbox}

\title{Demo of \texttt{macosbox}}
\author{Léo Valais}
\date\today

\usemintedstyle{colorful}
\usemacosboxmintedstyle{xcode}
\usemacosboxdarkmintedstyle{monokai}

\newminted{latex}{linenos}

\newcommand{\code}[1][Code]{\medskip\underline{\textit{\large{}#1:}}}

\begin{document}
\maketitle

\section{Light boxes}
\subsection{Simple box}
\begin{macosbox}{A macOS window-like \texttt{tcolorbox}}
  Aenean in sem ac leo mollis blandit.
  Pellentesque tristique imperdiet tortor.
  Donec vitae dolor.
\end{macosbox}

\code
\begin{latexcode}
\begin{macosbox}{A macOS window-like \texttt{tcolorbox}}
  Aenean in sem ac leo mollis blandit.
  Pellentesque tristique imperdiet tortor.
  Donec vitae dolor.
\end{macosbox}
\end{latexcode}

\subsection{Box containing code}
A \texttt{macosbox} can contain code with a different syntax highlighting style. For
instace, this document begins with:
\begin{latexcode}
\usemintedstyle{colorful}
\usemacosboxmintedstyle{xcode}
\end{latexcode}
It means that in this document, the \texttt{colorful} style is used but inside \texttt{macosbox}es,
the \texttt{xcode} style is used instead. \textit{Only \texttt{minted} is supported (not \texttt{listings} or any other).}

Default value: \texttt{xcode}.

\texttt{minted} is not a requirement for \texttt{macosbox}. If \texttt{minted} is
imported this feature will enable automatically.

\begin{macosbox}{A box containing code}
\begin{minted}{common-lisp}
(defun ! (n)
  (if (zerop n)
      0
      (* n (! (1- n)))))
\end{minted}
\end{macosbox}

\code
\begin{latexcode}
\begin{macosbox}{A box containing code}
\begin{minted}{common-lisp}
(defun ! (n)
  (if (zerop n)
      0
      (* n (! (1- n)))))
\end{minted}
\end{macosbox}
\end{latexcode}

\section{Dark boxes}
\subsection{Simple box}
\begin{macosdarkbox}{A dark box}
  This is a dark box with the colors of macOS Mojave.
\end{macosdarkbox}

\code
\begin{latexcode}
\begin{macosdarkbox}{A dark box}
  This is a dark box with the colors of macOS Mojave.
\end{macosdarkbox}
\end{latexcode}

\subsection{Dark box containing code}
Likewise, the \texttt{minted} theme inside dark boxes can be chosen using:

\begin{latexcode}
\usemacosboxdarkmintedstyle{monokai}
\end{latexcode}

Default value: \texttt{monokai}.

\begin{macosdarkbox}{A dark box containing code}
\begin{minted}{c}
#include <stdio.h>
#include <stdlib.h>

int main(void) {
  printf("hello, world\n");
  return EXIT_SUCCESS;
}
\end{minted}
\end{macosdarkbox}

\code
\begin{latexcode}
\begin{macosdarkbox}{A dark box containing code}
\begin{minted}{c}
#include <stdio.h>
#include <stdlib.h>

int main(void) {
  printf("hello, world\n");
  return EXIT_SUCCESS;
}
\end{minted}
\end{macosdarkbox}
\end{latexcode}

\section{About this package}
This package was originally designed to make popups in beamer presentations that
look like macOS windows.

Nevertheless, this package still works inside classic \LaTeX{} documents.

\end{document}
